%%------------------------------------------------------------------------------
%%Klasse
%%------------------------------------------------------------------------------
\documentclass[a5paper,landscape,DIV=calc,11pt,notitlepage]{scrartcl}
%%------------------------------------------------------------------------------
%%Packages
%%------------------------------------------------------------------------------
\usepackage[headsepline=no, footsepline=no]{scrlayer-scrpage}
\usepackage[utf8]{inputenc}
\usepackage[T1]{fontenc}
\usepackage{CJKutf8}
\usepackage[ngerman]{babel} 
\usepackage[left=36pt, hmarginratio=3:2, top=24pt, bottom=24pt,twoside=true]{geometry}
\usepackage{tabularx}
\usepackage[svgnames,RGB,table]{xcolor}
\usepackage{multicol}
\usepackage{booktabs}
\usepackage{multicol,multirow}
\usepackage[osf]{libertine}
\usepackage{microtype}
\usepackage[german]{layout}
\usepackage{tcolorbox}
\tcbuselibrary{raster,skins}
\usepackage{eso-pic}
\usepackage{graphicx}
\usepackage{amssymb}
\usepackage{amsmath}
\usepackage{mathtools}
\usepackage{setspace}
\usepackage{hyperref}
\hypersetup{colorlinks, citecolor=, linkcolor=, urlcolor=blue,pdftex,pdfauthor={Sascha Christmann},pdftitle={G\={o}j\={u}-Ry\={u} Karate-D\={o} Reusrath - Trainigskarten},pdfsubject={Trainingsmaterialien G\={o}j\={u}-Ry\={u} Karate-D\={o}},pdfproducer={MiKTeX},pdfcreator={pdflatex}}
\usepackage{datetime}
\usepackage{import}
\usepackage{listings}
\lstloadlanguages{[LaTeX]TeX}
\usepackage{soul}
%%------------------------------------------------------------------------------
%%Vereinbarungen & Makros
%%------------------------------------------------------------------------------
\clearpairofpagestyles
\newcounter{num}
\newcounter{numz}
\newcommand{\ctu}[0]{\stepcounter{num}\arabic{num}.}
\newcommand{\ctuz}[0]{\stepcounter{numz}\arabic{numz}}
\newcommand{\ctd}[0]{\addtocounter{num}{-1}\arabic{num}.}
\newcommand{\ctdz}[0]{\addtocounter{numz}{-1}\arabic{numz}}	
\setcounter{num}{0}
\setcounter{numz}{0}
%%------------------------------------------------------------------------------
\setuldepth{a}
%%------------------------------------------------------------------------------
\renewcommand\UrlFont{\color{blue}\rmfamily\itshape}
%%------------------------------------------------------------------------------
\renewcommand*{\familydefault}{\sfdefault}
%%------------------------------------------------------------------------------
\newlength\tindent
\setlength{\tindent}{\parindent}
\setlength{\parindent}{0pt}
\renewcommand{\indent}{\hspace*{\tindent}}
%%------------------------------------------------------------------------------
\graphicspath{{Gfx/gkd/}{Gfx/habersetzer/}{Gfx/gurte/}{Gfx/sglgrkr/}{Gfx/self/}{Gfx/yuishinkan_kamen/}}
%%------------------------------------------------------------------------------
\definecolor{GKD}{rgb}{0.53,0.086,0.102}
\definecolor{SGL}{rgb}{0,0.580,0.290}
\definecolor{SGL2}{rgb}{0,1,0}
\definecolor{YBELT}{rgb}{1,1,0.2}
\definecolor{OBELT}{rgb}{1,0.404,0}
\definecolor{GBELT}{rgb}{0.431,0.686,0.282}
\definecolor{BLBELT}{rgb}{0,0.408,1}
\definecolor{BRBELT}{rgb}{0.376,0.22,0.075}
\definecolor{BKBELT}{rgb}{0.078,0.071,0.071}
%%------------------------------------------------------------------------------
\let\svthefootnote\thefootnote
\newcommand\freefootnote[1]{\let\thefootnote\relax\footnotetext{#1}\let\thefootnote\svthefootnote}
%%------------------------------------------------------------------------------
\newcommand{\trenner}[1]{\vspace{{#1}pt}\hrule\relax\vspace{{#1}pt}}
%%------------------------------------------------------------------------------
\newlength{\VTL}
\newcommand{\SETVTL}[1]{\settowidth{\VTL}{{#1}}}
%%------------------------------------------------------------------------------
\newcommand\ShowBelt[1]{\put(248,-188){\parbox[b][\paperheight]{\paperwidth}{\vfill\centering\includegraphics[width=48pt,keepaspectratio]{#1}\vfill}}}
%%------------------------------------------------------------------------------
\newcommand\ShowLogo{\put(-248,-188){\parbox[b][\paperheight]{\paperwidth}{\vfill\centering\includegraphics[width=48pt,keepaspectratio]{GJRKDR_n.pdf}\vfill}}}
%%------------------------------------------------------------------------------
\newcommand\ShowDraft{\put(-256,172){\parbox[b][\paperheight]{\paperwidth}{\vfill\centering\includegraphics[width=72pt,keepaspectratio]{entwurf.pdf}\vfill}}}
%%------------------------------------------------------------------------------
\tcbset{width=\textwidth,height=\textheight,right=12pt,left=12pt,fonttitle=\bfseries}
\newtcolorbox{pfbox}{natural height,leftrule=9pt,rightrule=9pt,right=4pt,left=4pt,colback=white,colframe=GKD}
%%------------------------------------------------------------------------------
\newcommand{\zwitepf}{\begin{pfbox}
		Tori greift vorgehend aus Sanchin-Dachi an - der Angriff wird klar erkennbar und schulmäßig ausgeführt. Zu beachten sind gutes Distanzverhalten von Tori und Uke, die erkennbare Ernsthaftigkeit von Angriff und Konter sowie ein runder Ablauf. 	%%Kumite Ura https://www.youtube.com/watch?v=ED4-_aApmh4
		%%Nage Waza https://www.youtube.com/watch?v=TAB9bBKQpvQ
\end{pfbox}}
%%------------------------------------------------------------------------------
%%EOF
%%------------------------------------------------------------------------------