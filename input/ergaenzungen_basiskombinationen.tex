\setcounter{num}{0}
\setcounter{numz}{0}
\setlength{\tabcolsep}{3pt}		
\begin{tcolorbox}[width=\textwidth,height=\textheight,right=12pt,left=12pt,colframe=GKD,colback=white,fonttitle=\bfseries,coltitle=white,title=Allgemeines:\indent Ergänzungen / Basiskombinationen]
	\null\vfill\null
	\begin{tabularx}{\textwidth}{llXXX}
		\textbf{Stand} 		& & \textbf{Technik 1\,\(\odot\)} 	& \textbf{Technik 2\,\(\odot\)} 	& \textbf{Technik 3\,\(\odot\)}\\
		\midrule
		Sanchin Dachi 		& \(\hookleftarrow\)	& Age-Uke				& Kake-Uke 				& Harai-Otoshi-Uke	\\
		\textit{oder}		& \(\hookrightarrow\) 	& J\={o}dan-Zuki 		& Teisho-Zuki Gedan		& Ura-Zuki			\\
		Zenkutsu Dachi		& \(\hookleftarrow\)	& Yoko-Uke 				& Shuto-Uke 			& Gedan Teisho-Uke	\\
		\textit{oder}		& \(\hookrightarrow\)	& Ch\={u}dan-Zuki 		& Nukite (Kehle) 		& Uraken-Uchi		\\
		Shiko Dachi			& \(\hookleftarrow\)	& Soto-Uke 				& Mawashi-Uke 			& Haishu-Uke		\\
		& \(\hookrightarrow\)	& Ch\={u}dan-Zuki	& Teisho-Ate 			& Furi-Uchi			\\
		\midrule
		\textbf{Stand} 		& & \textbf{Technik\,\(\odot\)} 		&  						&\\
		\midrule
		Sanchin Dachi 		& \(\hookrightarrow\)	& Oi-Zuki J\={o}dan 	&  						&\\
		Zenkutsu Dachi		& \(\hookrightarrow\)	& Oi-Zuki Ch\={u}dan 	&\multicolumn{2}{l}{Nach Abschluss dann}\\
		Shiko Dachi			& \(\hookrightarrow\)	& Oi-Zuki Gedan 		&\multicolumn{2}{l}{seitengedreht von oben weiter}\\
		Shiko Dachi			& \(\hookleftarrow\)	& Harai-Otoshi-Uke 		&  						&\\
		Zenkutsu Dachi		& \(\hookleftarrow\)	& Yoko-Uke 				&  						&\\
		Sanchin Dachi		& \(\hookleftarrow\)	& Age-Uke 				&  						&\\
	\end{tabularx}\\\null\vfill\null
	\setlength{\tabcolsep}{6pt}	
	\begin{center}
		\parbox{\textwidth-2\tabcolsep}{Koordinationsübungen im Kihon. Im Block oben können Stände und Techniken beliebig gemischt werden. Der untere Block ist zur Übung der Fußbewegungen \textquotedblleft Tai Sabaki\textquotedblright~in der Vorwärts-, bzw. Rückwärtsbewegung gedacht.}
	\end{center}\null\vfill\null
\end{tcolorbox}
%%------------------------------------------------------------------------------
\clearpage
\pagebreak
%%------------------------------------------------------------------------------
\setcounter{num}{0}
\setcounter{numz}{0}
\setlength{\tabcolsep}{3pt}	
\begin{tcolorbox}[width=\textwidth,height=\textheight,right=12pt,left=12pt,colframe=GKD,colback=white,fonttitle=\bfseries,coltitle=white,title=Allgemeines:\indent Ergänzungen / Kombinationen / Koordination]
	\null\vfill\null
	\begin{tabularx}{\textwidth}{lllllllll}
		\textbf{Stand} 		& & \textbf{Technik 1\,\(\odot\)} 	&& \textbf{Technik 2\,\(\odot\)} 	&& \textbf{Technik 3\,\(\odot\)}&& \textbf{Technik 4\,\(\odot\)}\\
		\midrule
		Zenkutsu Dachi 		& \(\hookrightarrow\)	& Age-Uke					& \(\hookrightarrow\)	& Teisho-Zuki J\={o}dan	& \(\hookrightarrow\)	& Uraken-Uchi&&	\\
		\midrule
		Zenkutsu Dachi 		& \(\hookrightarrow\)	& Kake-Uke					& \(\hookrightarrow\)	& Ura-Zuki J\={o}dan	& \(\hookrightarrow\)	& Gyaku-Zuki Ch\={u}dan&&	\\
		\midrule
		Shiko Dachi			& \(\hookrightarrow\)	& Haito-Uke					& \(\downarrow\)		& Zenkutsu Dachi 		& \(\downarrow\)		& Nukite&&\\
		\midrule
		Zenkutsu Dachi		& \(\hookrightarrow\)	& Yoko-Uke					& \(\downarrow\)		& Tettsui-Uchi			& \(\hookrightarrow\)	& Mae-Geri &\(\downarrow\) & Gyaku-Zuki\\
		\midrule
		Sanchin Dachi		& \(\downarrow\)		& Harai-Otoshi-Uke			& \(\hookrightarrow\)	& Suri-Ashi Dachi		& \(\downarrow\)		& Ren-Zuki &\(\downarrow\) & Sanchin Dachi\\
		\midrule
		Shiko Dachi			& \(\hookrightarrow\)	& Empi-Uchi					& \(\downarrow\)		& Gyaku-Nukite			&&&&\\
		\midrule
		Neko-Ashi-Dachi		& \(\hookrightarrow\)	& Mawashi-Geri	& \(\downarrow\)		& Gyaku-Shuto-Uke		&\multicolumn{4}{l}{\textit{{\footnotesize Mawashi-Geri aus der Bewegung heraus}}}\\
		\midrule
		Zenkutsu Dachi		& \(\downarrow\)		& Harai-Otoshi-Uke			& \(\hookrightarrow\)	& Mae-Tobi-Geri			& \(\hookrightarrow\)	& Gyaku-Zuki &&\\
		\midrule
		Sanchin Dachi		& \(\hookrightarrow\)	& Yoko-Uke					& \(\downarrow\)		& Gyaku-Zuki			& \(\hookrightarrow\)	& Kake-Uke	& \(\downarrow\)& Teisho-Ate \\					
	\end{tabularx}\\\null\vfill\null
	\setlength{\tabcolsep}{6pt}	
\end{tcolorbox}
%%------------------------------------------------------------------------------
\clearpage
\pagebreak
%%------------------------------------------------------------------------------
\setcounter{num}{0}
\setcounter{numz}{0}	
\begin{tcolorbox}[width=\textwidth,height=\textheight,right=12pt,left=12pt,colframe=GKD,colback=white,fonttitle=\bfseries,coltitle=white,title=Allgemeines:\indent Ergänzungen / Kombinationen aus Kata]
	\null\vfill\null
	\begin{tabularx}{\textwidth}{lllllll}
		\multicolumn{7}{l}{\textbf{Sequenz aus Taikyoku Kake-Uke}}\\
		\midrule
		Sanchin Dachi 	& \(\hookrightarrow\) & Yoko-Uke	&\multicolumn{4}{l}{\(\odot\)}\\
		Sanchin Dachi 	& \(\hookrightarrow\) & Kake-Uke	&\multicolumn{4}{l}{\(\odot\)}\\
		Zenkutsu Dachi	& \(\hookrightarrow\) & Mae-Geri	& \(\odot\) & Empi-Uchi\,\(\odot\)&&\\
		\midrule
		\multicolumn{7}{l}{\textbf{Sequenz aus Gekisai Dai-Ichi und Gekisai Dai-Ni\quad}\textit{\scriptsize Kake Uke für Gekisai Dai-Ni}}\\
		\midrule	
		Sanchin Dachi 	& \(\hookrightarrow\) 	& Yoko-Uke	&\multicolumn{3}{l}{\(\odot\)} &\\
		Sanchin Dachi 	& \(\hookrightarrow\) 	& Yoko-Uke	&\multicolumn{3}{l}{\(\odot\)} &\\
		Zenkutsu Dachi 	& \(\hookrightarrow\) 	& Mae-Geri	&\(\odot\)	& \multicolumn{3}{l}{\(\downarrow\)\,Empi-Uchi\,\(\odot\)\,Uraken-Uchi\,\(\odot\)\,Harai-Otoshi-Uke\,\(\odot\)\,Gyaku-Zuki\,\(\odot\)}\\
		\midrule							
	\end{tabularx}\\\null\vfill\null
\end{tcolorbox}
%%------------------------------------------------------------------------------
\clearpage
\pagebreak
%%------------------------------------------------------------------------------
\setcounter{num}{0}
\setcounter{numz}{0}	
\begin{tcolorbox}[width=\textwidth,height=\textheight,right=12pt,left=12pt,colframe=GKD,colback=white,fonttitle=\bfseries,coltitle=white,title=Allgemeines:\indent B\={o}gy\={o} R\={o}k\={u} Kyod\={o} aus dem Japan Karate-D\={o} Jinen-Kai]
	\null\vfill\null
	\begin{center}
		\parbox{\textwidth-2\tabcolsep}{Es werden alle 6 Stände \& Techniken am Stück durchgeführt, in der Art, das immer der jeweils vordere Arm blockt (bzw. für die Fußtechniken der vordere Fuß kickt) und alle Standwechsel lediglich über den vorderen Fuß erfolgen. Als erste Übungsform wird nur der Block ausgeführt, als erste Steigerungsform wird der Konter hinzugenommen, als letzte Steigerung dann \mbox{Block \(\Rightarrow\)\,Kick \(\Rightarrow\)\,Konter.}}
	\end{center}
	\begin{tabularx}{\textwidth}{lllXlXll}
		%\multicolumn{7}{l}{\textbf{Konishi}}\\
		Stand	&&Block	&&Konter	&&Kick&\\
		\midrule
		Zenkutsu Dachi 	& \(\hookrightarrow\) & Age-Uke	&&Gyaku-Zuki	&&Mae-Geri&\textbf{Konishi Form}\\
		Shiko Dachi 	& \(\circledast\) & Harai-Otoshi-Uke	&&Kagi-Zuki&&Yoko-Geri&\\
		Zenkutsu Dachi	& \(\circledast\) & Yoko-Uke	&&Gyaku-Zuki&&Mae-Geri&\\
		Zenkutsu Dachi	& \(\circledast\) & Soto-Uke	&&Gyaku-Zuki&&Mae-Geri&\\
		Neko-Ashi-Dachi	& \(\circledast\) & Shuto Uke	&&Nukite&&Kansetsu-Geri&\\
		Shiko Dachi	& \(\hookleftarrow\) & Haito Uke	&&Gyaku-Zuki&&Ashi-Barai& \(\odot\)\\
		\midrule
	\end{tabularx}\\\null\vfill\null
\end{tcolorbox}