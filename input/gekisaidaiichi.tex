	\null\vfill\null
%{\small 	\begin{tabularx}{\textwidth}{llllX}\end{tabularx}}\null\vfill\null
\begin{center}
	\begin{minipage}[t]{\textwidth-2\tabcolsep}
		{
%		\setlength\columnseprule{0.5pt}
%		\begin{multicols}{3}
		\begin{enumerate}
			\item Tori und Uke in Sanchin Dachi Yoko Uke
			\item Tori greift mit Oi-Zuki Jodan an
			\item Uke blockt mit Age-Uke und kontert
			\item Tori blockt und geht 45° rechts in Shiko-Dachi mit Gyku Zuki Gedan
			\item Uke geht mit rechts zurück in Shiko-Dachi mit Harai Otoshi Uke
		\end{enumerate}
%		\end{multicols}
%				\setlength\columnseprule{0pt}
		\begin{multicols}{3}
			\begin{enumerate}
				\item ayshjkldhjklasd
				\item ayshjkldhjklasd
				\item ayshjkldhjklasd
				\item ayshjkldhjklasd
				\item ayshjkldhjklasd
				\item ayshjkldhjklasd
				\item ayshjkldhjklasd
				\item ayshjkldhjklasd
				\item ayshjkldhjklasd
				\item ayshjkldhjklasd
				\item ayshjkldhjklasd
				\item ayshjkldhjklasd
				\item ayshjkldhjklasd
				\item ayshjkldhjklasd
				\item ayshjkldhjklasd
			\end{enumerate}
		\end{multicols}
		%\footnotesize Als grundlegende Partnerformen der Unterstufe sollten die Techniken mit, dem jeweiligen Können entsprechender, Geschwindigkeit und Definition gezeigt werden. Stete Übung bringt Schnelligkeit. Zu beachten sind die richtige Distanz zum Partner und die korrekte Reaktion auf wechselnde Abstände - Kontertechniken mit Kime setzen. Die Basisübungen sind in der Form \textit{Gerader Angriff - Gerader Konter} beschrieben, sollten aber auch in den Sh\={o}t\={o}kan Varianten mit jeweils 45°-Ausweichbewegungen geübt werden, sind aber nicht Prüfungsrelevant. Dann sind kleinere Anpassungen notwendig, bzw.\,möglich, die sich aus dem Übungsansatz ergeben. Kontertechniken können, und dürfen, entsprechend dem jeweiligen Leistungsstand variiert werden.}\onehalfspacing\singlespacing
		%{\footnotesize Die Partnerformen können als \textbf{\textit{Drill}} geübt werden, indem Tori 10 Techniken mit Vorwärtsbewegung ausführt. Am Ende dann Wechsel von Tori zu Uke und 10 Techniken in die andere Richtung. Bei der Ausführung als Drill ist der stetige Seitenwechsel zu beachten, sowie das jeweilig \textquotedblleft richtige Hikite ziehen\textquotedblright. Im Drill bietet es sich an, den Stand auf Sanchin-Dachi zu verkürzen, bzw. für \textit{Gedan Oi-Zuki} auf Shiko-Dachi zu verlängern.
		}
	\end{minipage}
\end{center}\null\vfill\null