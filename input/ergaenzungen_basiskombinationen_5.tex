\null\vfill\null
\begin{center}
	\parbox{\textwidth-2\tabcolsep}{Basiskombination für Zuki im Zenkutsu-Dachi. In Abwandlung zum referenzierten Video kann man nach Abschluss der ersten Sequenz über Mawate den Seiten-/Richtungswechsel vornehmen und den Rückweg gehen.}
\end{center}
\begin{tabularx}{\textwidth}{lllX}
	%\multicolumn{7}{l}{\textbf{Konishi}}\\
	Stand	&&Technik	&\\
	\midrule
	Zenkutsu-Dachi 	& \(\hookleftarrow\) & Harai-Otoshi-Uke	&\\
	Zenkutsu-Dachi 	& \(\hookrightarrow\) & Oi-Zuki J\={o}dan	&\\
	Zenkutsu-Dachi 	& \(\hookrightarrow\) & Ren-Zuki J\={o}dan \&~Ch\={u}dan	&\\
	Zenkutsu-Dachi 	& \(\hookrightarrow\) & Gyaku-Zuki Ch\={u}dan	&2x schnell vorgehen\\
	Zenkutsu-Dachi 	& \(\hookrightarrow\) & Gyaku-Zuki J\={o}dan	&\\
	Zenkutsu-Dachi 	& \(\hookrightarrow\) & Gyaku-Zuki Ch\={u}dan	&2x schnell vorgehen\\
	Zenkutsu-Dachi 	& \(\hookrightarrow\) & Harai-Otoshi-Uke	& Mawate \& seitengespiegelt von oben\\
	\midrule
\end{tabularx}\\\null\vfill\null