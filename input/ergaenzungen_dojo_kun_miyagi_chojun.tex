	\begin{footnotesize}
	\begin{tabularx}{\textwidth}{X}
		\textbf{Grundsätze des okinawanischen G\={o}j\={u}-Ry\={u}} \\
		\midrule
		Es sollte bekannt sein, dass geheime Prinzipien des G\={o}j\={u}-Ry\={u} in den Kata existieren. \\
		G\={o}j\={u}-Ry\={u} Karate-D\={o} ist eine Manifestation der harmonischen Übereinstimmung des Universums im eigenen Selbst. \\ 
		Der Weg des G\={o}j\={u}-Ry\={u} Karate-D\={o} besteht darin, den Weg der Tugend zu suchen.\\
	\end{tabularx}\null\vfill\null
	\begin{tabularx}{\textwidth}{Xr}	
		\hfill & \textbf{Geduld} \\
		\midrule
		\hfill & Du musst vor allem die Kunst der wahren und echten Geduld erlernen. \\
		\hfill & Folge dem Weg der Geduld bis zur siebten Kraft und sei nie in Eile zu lernen. \\
		\hfill & Denke immer zuerst nach und vermeide überstürztes Handeln. \\
		\hfill & Füge niemals jemandem Schaden zu oder lasse zu, dass man dir Schaden zufügt. \\
	\end{tabularx}\null\vfill\null
	\begin{tabularx}{\textwidth}{X}	
		\textbf{D\={o}j\={o}kun}\\
		\midrule
		Achte auf deine Höflichkeit mit Bescheidenheit. \\
		Trainiere dich in Anbetracht deiner körperlichen Stärke. \\
		Studiere und entwerfe ernsthaft. \\
		Sei ruhig im Geist und schnell im Handeln. \\
		Achte auf dich selbst. \\
		Lebe ein schlichtes und einfaches Leben. \\
		Sei nicht zu stolz auf dich. \\
		Übe weiter mit Geduld und Demut. \\
	\end{tabularx}\null\vfill\null
	\begin{tabularx}{\textwidth}{Xr}	
		\hfill & \textbf{Acht Gebote aus dem Bubishi}\\
		\midrule
		\hfill & Der Geist ist eins mit Himmel und Erde. \\
		\hfill & Der Kreislaufrhythmus des Körpers ist dem Zyklus von Sonne und Mond ähnlich. \\
		\hfill & Die Art des Ein- und Ausatmens ist hart und weich. \\
		\hfill & Handle im Einklang mit der Zeit und dem Wandel. \\
		\hfill & Die Techniken werden in Abwesenheit bewusster Gedanken ausgeführt. \\
		\hfill & Die Füße müssen sich vor- und zurückziehen, sich trennen und treffen. \\
		\hfill & Den Augen entgeht nicht einmal die kleinste Veränderung. \\
		\hfill & Die Ohren hören gut in alle Richtungen. \\
	\end{tabularx}
	\end{footnotesize}