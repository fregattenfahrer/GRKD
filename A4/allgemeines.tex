%%------------------------------------------------------------------------------
%%Allgemeines
%%------------------------------------------------------------------------------
	\setcounter{num}{0}
	\setcounter{numz}{0}
	\begin{tcolorbox}[width=\textwidth,right=12pt,left=12pt,colframe=lightgray,colback=white,fonttitle=\bfseries,coltitle=black,title=Allgemeines:\indent Trainingsablauf Begrüßung und Ende]
		\null\vfill\null	
		\begin{tabularx}{\textwidth}{llX}
			\multicolumn{2}{l}{\textbf{Japanisch}} 	& \textbf{Deutsch}\\
			\midrule
			\ctu		& Seiza 				& Kniesitz. Füße unter dem Gesäß, Hände liegen auf den Oberschenkeln\\
			\ctu		& Mokus\={o}			& Meditation\\
			\ctu		& Mokus\={o} Yame		& Ende der Meditation\\
			\ctu a		& Sh\={o}men ni Rei		& Gruß zur Vorderseite des D\={o}j\={o}\\
			\thenum .b	& Sensei ni Rei			& Gruß zum Meister\\
			\thenum .c	& Senpai ni Rei			& Gruß zum Fortgeschrittenen (als Trainer)\\
			\ctu		& Otagai ni Rei			& Gegenseitiger Gruß\\
			\ctu		& Kiritsu				& Aufstehen\\		
			\midrule
		\end{tabularx}\\\null\vfill\null
	\end{tcolorbox}
		\begin{center}
			\parbox{\textwidth-2\tabcolsep}{Zwischen 4\,\&\,5. kann, je nach D\={o}j\={o} ein \textit{\textquotedblleft onegaishimasu\textquotedblright}~zu Beginn, bzw.\,\textit{\mbox{\textquotedblleft arigat\={o} gozaimashita\textquotedblright}}~zum Ende des Trainings eingefügt werden. Zwischen 5.\,\&\,6. kann ein \textit{\textquotedblleft Ossu\textquotedblright}~eingefügt sein. Grundsätzlich läuft das Training in jedem D\={o}j\={o} nach diesem Schema ab und es ist wichtig, wenn man sich auf Lehrgängen befindet oder als Gast mittrainiert, sich nach der Etikette zu richten.}
		\end{center}\null\vfill\null
	
%%------------------------------------------------------------------------------
%%EOF
%%------------------------------------------------------------------------------